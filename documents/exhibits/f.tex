\subsubsection*{Safe Harbor Agreement for Whitehats - Frequently Asked Questions}\label{exhibit:f:faq_title}

This document ("FAQ") is meant to provide additional information to Protocol Communities about certain aspects of the Safe Harbor Agreement for Whitehats ("Agreement"). In the event of any conflict or inconsistency between the FAQ and the text of the Agreement, the text of the Agreement will govern. The information provided in the FAQ does not, and is not intended to, constitute legal advice.

\paragraph{Adoption and Initial Implementation}\label{exhibit:f:adoption}

\begin{enumerate}
    \item \textbf{What is a Protocol Community?} As defined in the Agreement, a Protocol Community is the set of key stakeholders with an interest in a blockchain-based protocol or similar decentralized technology. This group will typically include the DAO governing a protocol, DAO members and participants, protocol users, and any individual or group of individuals involved in securing the protocol.

    \item \textbf{If most blockchain-based protocols are meant to be open and permissionless, then how does a Protocol Community adopt the Agreement?} Exhibits B, C, and D to the Agreement provide guidance on how various groups can adopt the Agreement. Given that decentralized technologies are being developed, governed, and used in new and innovative ways, the Security Alliance recommends that Protocol Communities are thoughtful about how they publicize, deliberate about, and adopt the Agreement. Protocol Communities should consult with legal counsel about the adoption process as needed. Protocol Communities should consider the possibility that individuals and entities involved in developing, governing, and using their protocol may each use different communication channels to coordinate and discuss the protocol. Protocol Communities should consider using all of these channels to provide these individuals and entities with opportunities to learn about the Agreement, discuss its adoption, and agree to its terms. For instance, Protocol Communities may want to consider using popular communication channels, like Twitter and Discord, to publicize the Agreement. Protocol Communities might also coordinate with any independent entities that provide user interfaces for their protocol to engage with users. For some Protocol Communities, these steps may be helpful for promoting engagement with the Agreement adoption process.

    \item \textbf{Should the process of adopting the Agreement occur in public?} Yes, Protocol Communities are required by the Agreement to create an Agreement Fact Page that provides access to the materials associated with the adoption process. Protocol Communities should consider making all aspects of the adoption process public so that as many stakeholders as possible can engage with the process.

          % COMMENT: The original document contains references to "Agreement Fact Page" and "Adoption Form" but these terms are not defined in the main agreement. These may need clarification or definition.

    \item \textbf{What steps should a Protocol Community take to implement the program described in the Agreement?} As described in the Agreement, Protocol Communities should take the following steps to adopt the Agreement and implement the program described in it:

          \begin{enumerate}[label=\alph*.]
              \item Protocol Communities should clearly disclose the parameters selected during the DAO Adoption Procedures. The Agreement is an open-source template that requires Protocol Communities to add certain details and make certain decisions before it is adopted. These required items include:

                    \begin{enumerate}[label=\roman*.]
                        \item Specifying the protocol or other decentralized technology that will be subject to the Agreement. This process might require drafting a list of technical assets that are within the Agreement's scope;

                              % COMMENT: The original document references "Protocol Safety Address" but the main agreement uses "Asset Recovery Address". This inconsistency should be clarified.
                        \item Indicating the specific Protocol Safety Address where Whitehats will deposit assets that they recover;

                        \item Deciding whether to use a third-party vendor, like a bug bounty program administrator, to facilitate payment of the Bounty;

                        \item Deciding whether anonymous or pseudonymous Whitehats can participate in the program and collect a Bounty without identifying themselves to the Protocol Community. This decision will impact the extent to which the Protocol Community can perform diligence on the Whitehat in advance of their participation in the program or collection of the Bounty;

                        \item Deciding whether to perform sanctions diligence or other forms of diligence on Whitehats in advance of their participation in the program or their collection of the Bounty;

                        \item Deciding the percentage of Returnable Assets to be paid to Eligible Whitehats as a Bounty, which may involve reviewing the payment amounts associated with a Protocol Community's existing bug bounty program, if any; and

                        \item Deciding whether Whitehats will be permitted to deduct the Bounty themselves from the assets that they recover. This decision will limit the extent to which the Protocol Community can perform diligence on the Whitehat or assess their compliance with the Agreement before the Whitehat collects the Bounty.
                    \end{enumerate}

              \item Protocol Communities may consider making additional determinations which, if made, should also be included in both the Adopting Addendum and on the Adoption Form. These additional determinations may include:

                    % COMMENT: "Adopting Addendum" is referenced but not defined in the main agreement. This may need clarification.

                    \begin{enumerate}[label=\roman*.]
                        \item Deciding whether to impose any additional cap(s) on the Bounty paid in connection with an Urgent Blackhat Exploit, such as an aggregate cap equivalent to a US Dollar amount and above which payment will not be made to an Eligible Whitehat(s), or a fixed cap applicable to each Eligible Whitehat contributing to an Eligible Funds Rescue; and

                        \item Incorporating other due diligence requirements on Whitehats that address the unique needs of the Protocol Community adopting the Agreement.
                    \end{enumerate}

              \item Protocol Communities should consult with legal counsel in relevant jurisdictions about the specific legal risks and benefits of each of the choices described above because they may expose the Protocol Community or Protocol Community Members to legal or regulatory risk.

              \item As described above, Protocol Communities must make certain information about the adoption process publicly accessible. Protocol Communities should consider taking other steps to include different stakeholders in the process.

              \item Protocol Communities should also consider communicating to potential Whitehats whether there are any limits to the release provisions provided by the Agreement based on the Protocol Community's specific circumstances. For instance, a Protocol Community might take the position that the Agreement does not bind the Protocol's Users or other Protocol Community Members. Under that circumstance, the Protocol Community should consider notifying Whitehats that the release provisions might not protect them from claims brought by persons or entities who are not parties to the Agreement.

              \item Protocol Communities should consider the additional steps needed to implement the program. These steps may include, but are not limited to, coordinating with a bug bounty program administrator and creating internal organizational structures for administering the program.
          \end{enumerate}
\end{enumerate}

\paragraph{Compliance with Applicable Laws and Regulations}\label{exhibit:f:compliance}

\begin{enumerate}
    \item \textbf{How can Protocol Communities adapt the Agreement so that it complies with applicable laws and regulations?} The Agreement is a template agreement that is meant to be adapted for use by sophisticated Protocol Communities around the world. Protocol Communities are encouraged to customize the parameters of the Agreement via the DAO Adoption Procedures parameter settings so that it conforms with the specific laws and regulations that apply to them and otherwise meets their particular needs.

    \item \textbf{Should Protocol Communities take any steps to ensure that their Bounty payments to Whitehats comply with international sanctions regimes?} Yes, each participating Protocol Community is expected to comply with applicable sanctions obligations, and the Security Alliance recommends that Protocol Communities implement a risk-based approach to ensuring compliance with these obligations. For example, while Section \ref{subsec:money_laundering} of the Agreement requires Whitehats to represent that they are not subject to any national or international sanctions regimes, in some jurisdictions, risk of sanctions violations may be increased where Whitehats are able to anonymously attempt an Eligible Funds Rescue and receive or retain Returnable Assets as a Bounty. This risk may also be heightened where the Protocol Community does not take other steps, such as conducting pre-payment diligence and instituting monitoring measures, to prevent payment to a sanctioned entity. The Security Alliance further recommends that Protocol Communities consult with legal counsel about how to address potential risks associated with the applicable sanctions regime(s) and to discuss what measures Protocol Communities may wish to take to comply with the applicable regime(s).

    \item \textbf{Should Protocol Communities make Whitehats aware of the risks associated with the Agreement and the program that it describes?} Yes. The Agreement includes a list of risk disclosures in Exhibit E. Protocol Communities should consider adding or modifying those risk disclosures to account for any risks that are specific to their situation. These specific risks might address positions that law enforcement or regulators may take with respect to the program in particular jurisdictions. Protocol Communities should consult with legal counsel about these risks as needed.
\end{enumerate}